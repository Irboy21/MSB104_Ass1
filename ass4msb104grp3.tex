% Options for packages loaded elsewhere
% Options for packages loaded elsewhere
\PassOptionsToPackage{unicode}{hyperref}
\PassOptionsToPackage{hyphens}{url}
\PassOptionsToPackage{dvipsnames,svgnames,x11names}{xcolor}
%
\documentclass[
  letterpaper,
  DIV=11,
  numbers=noendperiod]{scrartcl}
\usepackage{xcolor}
\usepackage{amsmath,amssymb}
\setcounter{secnumdepth}{-\maxdimen} % remove section numbering
\usepackage{iftex}
\ifPDFTeX
  \usepackage[T1]{fontenc}
  \usepackage[utf8]{inputenc}
  \usepackage{textcomp} % provide euro and other symbols
\else % if luatex or xetex
  \usepackage{unicode-math} % this also loads fontspec
  \defaultfontfeatures{Scale=MatchLowercase}
  \defaultfontfeatures[\rmfamily]{Ligatures=TeX,Scale=1}
\fi
\usepackage{lmodern}
\ifPDFTeX\else
  % xetex/luatex font selection
\fi
% Use upquote if available, for straight quotes in verbatim environments
\IfFileExists{upquote.sty}{\usepackage{upquote}}{}
\IfFileExists{microtype.sty}{% use microtype if available
  \usepackage[]{microtype}
  \UseMicrotypeSet[protrusion]{basicmath} % disable protrusion for tt fonts
}{}
\makeatletter
\@ifundefined{KOMAClassName}{% if non-KOMA class
  \IfFileExists{parskip.sty}{%
    \usepackage{parskip}
  }{% else
    \setlength{\parindent}{0pt}
    \setlength{\parskip}{6pt plus 2pt minus 1pt}}
}{% if KOMA class
  \KOMAoptions{parskip=half}}
\makeatother
% Make \paragraph and \subparagraph free-standing
\makeatletter
\ifx\paragraph\undefined\else
  \let\oldparagraph\paragraph
  \renewcommand{\paragraph}{
    \@ifstar
      \xxxParagraphStar
      \xxxParagraphNoStar
  }
  \newcommand{\xxxParagraphStar}[1]{\oldparagraph*{#1}\mbox{}}
  \newcommand{\xxxParagraphNoStar}[1]{\oldparagraph{#1}\mbox{}}
\fi
\ifx\subparagraph\undefined\else
  \let\oldsubparagraph\subparagraph
  \renewcommand{\subparagraph}{
    \@ifstar
      \xxxSubParagraphStar
      \xxxSubParagraphNoStar
  }
  \newcommand{\xxxSubParagraphStar}[1]{\oldsubparagraph*{#1}\mbox{}}
  \newcommand{\xxxSubParagraphNoStar}[1]{\oldsubparagraph{#1}\mbox{}}
\fi
\makeatother


\usepackage{longtable,booktabs,array}
\usepackage{calc} % for calculating minipage widths
% Correct order of tables after \paragraph or \subparagraph
\usepackage{etoolbox}
\makeatletter
\patchcmd\longtable{\par}{\if@noskipsec\mbox{}\fi\par}{}{}
\makeatother
% Allow footnotes in longtable head/foot
\IfFileExists{footnotehyper.sty}{\usepackage{footnotehyper}}{\usepackage{footnote}}
\makesavenoteenv{longtable}
\usepackage{graphicx}
\makeatletter
\newsavebox\pandoc@box
\newcommand*\pandocbounded[1]{% scales image to fit in text height/width
  \sbox\pandoc@box{#1}%
  \Gscale@div\@tempa{\textheight}{\dimexpr\ht\pandoc@box+\dp\pandoc@box\relax}%
  \Gscale@div\@tempb{\linewidth}{\wd\pandoc@box}%
  \ifdim\@tempb\p@<\@tempa\p@\let\@tempa\@tempb\fi% select the smaller of both
  \ifdim\@tempa\p@<\p@\scalebox{\@tempa}{\usebox\pandoc@box}%
  \else\usebox{\pandoc@box}%
  \fi%
}
% Set default figure placement to htbp
\def\fps@figure{htbp}
\makeatother


% definitions for citeproc citations
\NewDocumentCommand\citeproctext{}{}
\NewDocumentCommand\citeproc{mm}{%
  \begingroup\def\citeproctext{#2}\cite{#1}\endgroup}
\makeatletter
 % allow citations to break across lines
 \let\@cite@ofmt\@firstofone
 % avoid brackets around text for \cite:
 \def\@biblabel#1{}
 \def\@cite#1#2{{#1\if@tempswa , #2\fi}}
\makeatother
\newlength{\cslhangindent}
\setlength{\cslhangindent}{1.5em}
\newlength{\csllabelwidth}
\setlength{\csllabelwidth}{3em}
\newenvironment{CSLReferences}[2] % #1 hanging-indent, #2 entry-spacing
 {\begin{list}{}{%
  \setlength{\itemindent}{0pt}
  \setlength{\leftmargin}{0pt}
  \setlength{\parsep}{0pt}
  % turn on hanging indent if param 1 is 1
  \ifodd #1
   \setlength{\leftmargin}{\cslhangindent}
   \setlength{\itemindent}{-1\cslhangindent}
  \fi
  % set entry spacing
  \setlength{\itemsep}{#2\baselineskip}}}
 {\end{list}}
\usepackage{calc}
\newcommand{\CSLBlock}[1]{\hfill\break\parbox[t]{\linewidth}{\strut\ignorespaces#1\strut}}
\newcommand{\CSLLeftMargin}[1]{\parbox[t]{\csllabelwidth}{\strut#1\strut}}
\newcommand{\CSLRightInline}[1]{\parbox[t]{\linewidth - \csllabelwidth}{\strut#1\strut}}
\newcommand{\CSLIndent}[1]{\hspace{\cslhangindent}#1}



\setlength{\emergencystretch}{3em} % prevent overfull lines

\providecommand{\tightlist}{%
  \setlength{\itemsep}{0pt}\setlength{\parskip}{0pt}}



 


\KOMAoption{captions}{tableheading}
\makeatletter
\@ifpackageloaded{caption}{}{\usepackage{caption}}
\AtBeginDocument{%
\ifdefined\contentsname
  \renewcommand*\contentsname{Table of contents}
\else
  \newcommand\contentsname{Table of contents}
\fi
\ifdefined\listfigurename
  \renewcommand*\listfigurename{List of Figures}
\else
  \newcommand\listfigurename{List of Figures}
\fi
\ifdefined\listtablename
  \renewcommand*\listtablename{List of Tables}
\else
  \newcommand\listtablename{List of Tables}
\fi
\ifdefined\figurename
  \renewcommand*\figurename{Figure}
\else
  \newcommand\figurename{Figure}
\fi
\ifdefined\tablename
  \renewcommand*\tablename{Table}
\else
  \newcommand\tablename{Table}
\fi
}
\@ifpackageloaded{float}{}{\usepackage{float}}
\floatstyle{ruled}
\@ifundefined{c@chapter}{\newfloat{codelisting}{h}{lop}}{\newfloat{codelisting}{h}{lop}[chapter]}
\floatname{codelisting}{Listing}
\newcommand*\listoflistings{\listof{codelisting}{List of Listings}}
\makeatother
\makeatletter
\makeatother
\makeatletter
\@ifpackageloaded{caption}{}{\usepackage{caption}}
\@ifpackageloaded{subcaption}{}{\usepackage{subcaption}}
\makeatother
\usepackage{bookmark}
\IfFileExists{xurl.sty}{\usepackage{xurl}}{} % add URL line breaks if available
\urlstyle{same}
\hypersetup{
  pdftitle={Økonomisk vekst og regional ulikhet: En empirisk analyse av sammenhengen mellom vekst og Gini-koeffisient på tvers av regioner},
  pdfauthor={Irjan \& Magnus},
  colorlinks=true,
  linkcolor={blue},
  filecolor={Maroon},
  citecolor={Blue},
  urlcolor={Blue},
  pdfcreator={LaTeX via pandoc}}


\title{Økonomisk vekst og regional ulikhet: En empirisk analyse av
sammenhengen mellom vekst og Gini-koeffisient på tvers av regioner}
\author{Irjan \& Magnus}
\date{10 December 2025}
\begin{document}
\maketitle


\subsection{}\label{section}

Irjan tar oddetall. Magnum opus tar partall

\subsection{Objective}\label{objective}

\subsection{Synthesize and consolidate the critical findings from
Assignments 1, 2, and 3 into a
coherent}\label{synthesize-and-consolidate-the-critical-findings-from-assignments-1-2-and-3-into-a-coherent}

academic paper. Max size 20 pages.

\subsection{0. Short abstract}\label{short-abstract}

\subsection{I. Introduction}\label{i.-introduction}

• Background:~ Provide a brief background on the topics of regional
development and inequality.

Objectives:~Outline the main objectives and research questions addressed
in Assignments 1, 2, and 3. • Significance:~Highlight the importance and
relevance of the research undertaken.

\subsection{II.Literaturgjennomgang}\label{ii.literaturgjennomgang}

(Piketty, 2017) Previous Work:~Summarize key literature related to
regional development and inequality. ‣ Utilize the RStudio citation
tools in your summary of the literature. • Research Gap:~Identify the
research gap your assignments aim to address.

Forholdet mellom økonomisk vekst og inntektsulikhet har lenge vært et
sentralt og omdiskutert tema i økonomisk teori. Den klassiske
Kuznets-hypotesen (Kuznets, 1955) foreslår en omvendt U-formet kurve
hvor ulikheten stiger i tidlige utviklingsfaser for så å falle når
økonomien modnes. Selv om dette rammeverket har sterk historisk
innflytelse, har senere forskning vist at mekanismene bak vekst og
ulikhet er langt mer komplekse, og at sammenhengen varierer betydelig
mellom regioner og tidsperioder (Piketty, 2017).

En viktig utvidelse kommer fra moderne regionaløkonomisk teori, særlig
forskning på konvergens og regional divergens. Barro \& Sala-i-Martin
(2004) argumenterer for at regioner kan nærme seg hverandre økonomisk
(konvergere) dersom kapital og arbeidskraft flyter fritt, men at
ulikheter kan vedvare når særtrekk som utdanningsnivå, innovasjonsevne
eller næringsstruktur skiller regionene.

Et sentralt bidrag til denne litteraturen er Lessmann (2017), som
undersøker regional ulikhet og konvergens i et globalt perspektiv ved
hjelp av satellittdata. Denne studien viser at regional ulikhet ikke
nødvendigvis avtar i takt med økonomisk utvikling, og at forskjeller i
institusjoner, politisk stabilitet og geografiske forhold spiller en
langt større rolle enn tidligere antatt. Lessmann finner også at
regioner med svakere styringskvalitet og lavere humankapitalutvikling
har en tendens til å oppleve mer vedvarende eller økende ulikhet. Dette
støtter tanken om at vekst alene ikke er tilstrekkelig for å redusere
ulikhet uten at underliggende strukturelle faktorer tas i betraktning.

I nyere litteratur samler Mdingi \& Ho (2021) et bredt spekter av
studier om sammenhengen mellom inntektsulikhet og økonomisk vekst. Deres
gjennomgang understreker at relasjonen mellom vekst og ulikhet ikke er
entydig: enkelte studier finner at ulikhet hemmer vekst, andre at vekst
øker ulikhet, og mange at sammenhengen varierer med landets
utviklingsnivå, institusjoner og kredittmarkedets funksjon.

Det finnes også gode teoretiske argumenter for at økonomisk vekst kan
bidra til økt ulikhet. Dette gjelder særlig i regioner der avkastningen
på kapital er høy, mens investeringer i kompetanse og humankapital
henger etter. I slike områder vil gevinster fra vekst lettere tilfalle
dem som allerede er økonomisk sterke. Nettopp derfor er det viktig å
analysere sammenhengen mellom vekst og ulikhet med et empirisk opplegg
som tar hensyn til både økonomiske forskjeller, utdanningsnivå og
institusjonelle forhold.

Samlet sett peker litteraturen mot tre sentrale innsikter:

\begin{itemize}
\item
  \textbf{Vekst kan både øke og redusere ulikhet}, avhengig av regionens
  utviklingsnivå og institusjonelle rammeverk.
\item
  \textbf{Utdanning, arbeidsledighet og styringskvalitet} er viktige
  modererende faktorer som påvirker hvordan vekst omsettes i økonomiske
  gevinster.
\item
  \textbf{Regionale forskjeller er avgjørende}, og moderne empirisk
  forskning (som Lessmann, 2017) viser at ulikhet ofte vedvarer selv
  under økonomisk vekst --- særlig i regioner med svakere institusjoner
  eller lavere humankapital.
\end{itemize}

Denne teorien gir et solid grunnlag for å tolke resultatene i den
empiriske delen av oppgaven.

\subsubsection{Reaserchgap}\label{reaserchgap}

Til tross for omfattende forskning gjenstår flere viktige kunnskapshull.
For det første undersøker mange studier sammenhengen mellom økonomisk
vekst og ulikhet på nasjonalt nivå, mens det finnes færre analyser som
fokuserer på regioner innenfor samme land, der forskjeller i demografi,
kompetansenivå og arbeidsmarked kan ha stor betydning. Dette er en
viktig mangel, ettersom nasjonale gjennomsnittstall ofte skjuler interne
forskjeller.

For det andre er det begrenset med forskning som analyserer hvordan
regionale kjennetegn modererer effekten av vekst på ulikhet. Teorier
peker på at utdanningsnivå, arbeidsledighet og institusjonelle forhold
kan påvirke hvordan vekst omsettes i inntektsfordeling, men empiriske
tester av slike mekanismer er relativt få, spesielt på subnasjonalt
nivå.

For det tredje er det stor variasjon i metodiske tilnærminger. Mange
studier analyserer kun en enkelt modell, uten å kombinere enkel
regresjon, undergruppeanalyser og interaksjonsmodeller i samme studie.
Dermed mangler det forskning som undersøker både den generelle
sammenhengen og hvordan den varierer mellom regioner med ulike
strukturelle egenskaper.

\subsection{III Data and Methodology}\label{iii-data-and-methodology}

• Data Sources:~Briefly describe the primary data sources used in the
previous assignments, noting any limitations. • Methodological
Approach:~Summarize the methodologies applied in Assignments 1, 2, and
3, highlighting the rationale behind choosing each method.

\subsection{IV.Empirical Findings}\label{iv.empirical-findings}

Analysene startet med en tverrsnittlig regresjon for å undersøke
sammenhengen mellom økonomisk vekst og inntektsulikhet. Den
grunnleggende modellen viste en positiv og statistisk signifikant effekt
av vekst, med en koeffisient på \textbf{0.538} (p \textless{} 0.01).
Modellen forklarte om lag \textbf{30 prosent} av variasjonen i ulikhet
(R² = 0.295). Da modellen ble utvidet med arbeidsledighet,
befolkningstetthet og utdanningsnivå, ble veksteffekten redusert til
\textbf{0.379} (p = 0.068). De strukturelle variablene hadde små og
ikke-signifikante effekter: arbeidsledighet \textbf{--0.0024},
befolkningstetthet \textbf{0.000022} og utdanningsnivå
\textbf{--0.0017}.

Når regionene ble delt etter utdanningsnivå, var veksteffekten
\textbf{--6.992} for regioner med lav utdanning, mens interaksjonsleddet
for middels utdanningsnivå var \textbf{7.547}, noe som ga en samlet
effekt på omtrent \textbf{0.56}. Ingen av estimatene var statistisk
signifikante.

Modellen som delte regionene etter arbeidsledighetsnivå viste tydeligere
mønstre. Veksteffekten var \textbf{0.37} i regioner med lav
arbeidsledighet og \textbf{0.38} i regioner med middels arbeidsledighet,
begge uten signifikant utslag. I regioner med høy arbeidsledighet var
derimot veksteffekten \textbf{--4.48}, og interaksjonsleddet
\textbf{--4.85}; begge koeffisientene var statistisk signifikante.

Befolkningstetthet ga svakere resultater. Veksteffekten var
\textbf{0.376} i regioner med lav tetthet, økte til \textbf{0.88} i
regioner med middels tetthet og falt til \textbf{0.26} i regioner med
høy tetthet. Ingen av disse estimatene, verken hovedleddene eller
interaksjonsleddene, var statistisk signifikante.

Til slutt ble det testet ikke-lineære funksjonelle former. Den
logaritmiske modellen viste et negativt og statistisk signifikant
interaksjonsledd mellom log(vekst) og høy arbeidsledighet
(\textbf{--0.115}). Den kvadratiske modellen hadde den høyeste
forklaringskraften av alle spesifikasjonene (R² ≈ \textbf{0.51}). I
denne modellen var interaksjonsleddet mellom vekst² og høy
arbeidsledighet \textbf{--124.74}, også statistisk signifikant. Ingen av
de ikke-lineære spesifikasjonene viste signifikante effekter for
utdanningsnivå eller befolkningstetthet.

\subsection{V.Diskusjon}\label{v.diskusjon}

Resultatene fra analysene viser at forholdet mellom økonomisk vekst og
inntektsulikhet varierer betydelig mellom regioner, og at en enkel
lineær modell ikke gir et fullstendig bilde av denne sammenhengen. Den
første tverrsnittsanalysen viste en tydelig positiv og signifikant
sammenheng: regioner med høyere vekst hadde også høyere ulikhet. Dette
stemmer med etablerte teorier som hevder at økonomisk vekst i
utgangspunktet kan forsterke forskjeller fordi grupper med høyere
kompetanse eller kapital ofte drar større nytte av nye økonomiske
muligheter.

Når modellen utvides til å kontrollere for regionale strukturelle
forhold, og når interaksjonseffekter inkluderes, blir bildet langt mer
nyansert. Arbeidsledighet fremstår som den mest betydningsfulle
moderatorvariabelen. I regioner med høy arbeidsledighet er veksteffekten
både negativ og statistisk signifikant, noe som betyr at vekst faktisk
reduserer ulikhet i disse områdene. Dette funnet står i kontrast til
hovedresultatet fra tverrsnittet, og tyder på at i regioner med
betydelig arbeidsmarkedspress kan vekst bidra til å inkludere grupper
som tidligere sto utenfor arbeidsmarkedet. Sysselsettingseffekter kan
dermed være en sentral mekanisme som forklarer hvorfor ulikheten faller
når veksten øker i slike regioner.

For regioner med lav eller middels arbeidsledighet er effekten av vekst
svak og ikke signifikant. Det indikerer at når arbeidsmarkedet allerede
fungerer relativt godt, fordeles vekstgevinster på en måte som ikke
endrer ulikhetsnivået i særlig grad. Dette understreker at
arbeidsmarkedets tilstand er en nøkkelfaktor i hvordan vekst påvirker
inntektsfordelingen.

Utdanningsnivå og befolkningstetthet viser ingen statistisk sikre
effekter, selv om estimatene antyder enkelte forskjeller mellom grupper.
Dette kan bety at disse variablene har mer indirekte eller langsiktige
effekter som ikke fanges opp i det relativt begrensede datamaterialet.
Det kan også skyldes at variasjonen mellom regionene er for liten til å
identifisere tydelige mønstre. Likevel er det verdt å merke seg at slike
strukturelle variabler ofte virker i samspill med andre forhold -- som
næringssammensetning, teknologisk utvikling og institusjonelle rammer --
som ikke inngår i analysen.

De ikke-lineære modellene gir ytterligere støtte til at vekst ikke
virker likt i alle regioner. Både log-modellen og den kvadratiske
modellen viser at i regioner med høy arbeidsledighet blir den negative
effekten av vekst sterkere ved høyere vekstnivåer. Dette betyr at ikke
bare \emph{hvor} veksten skjer, men også \emph{hvor rask} den er, kan
påvirke hvordan inntektene fordeles. En slik dynamikk er vanskelig å
fange med en enkel lineær modell, og resultatene illustrerer derfor
viktigheten av å teste alternative spesifikasjoner når man studerer
fordelingsmessige utfall.

Disse funnene har klare implikasjoner for regional utviklingspolitikk.
For det første viser analysene at vekst alene ikke nødvendigvis
reduserer ulikhet. I regioner med lav arbeidsledighet kan vekst til og
med bidra til høyere ulikhet dersom gevinsten først og fremst tilfaller
grupper som allerede har sterke posisjoner i økonomien. Dette peker på
behovet for politikk som sikrer at vekst fordeles bredere, for eksempel
gjennom oppkvalifiseringstiltak, bedre tilgang til arbeid for
lavinntektsgrupper og strategier som øker mobilitet i arbeidsmarkedet.

For regioner med høy arbeidsledighet peker resultatene i motsatt
retning: her kan vekst fungere som et effektivt virkemiddel for å
redusere ulikhet. Dette betyr at vekstfremmende tiltak kan ha en
dobbeltgevinst i slike regioner, både ved å styrke den økonomiske
aktiviteten og ved å bidra til en mer rettferdig fordeling.
Investeringer i næringsutvikling, infrastruktur og arbeidsmarkedstiltak
kan derfor ha særlig stor betydning nettopp i regioner som sliter med
vedvarende arbeidsledighet.

Samlet sett viser analysene at økonomisk vekst ikke har én bestemt
effekt på ulikhet. Effekten avhenger i stor grad av regionenes
strukturelle forutsetninger, og spesielt av arbeidsmarkedets kapasitet
og fleksibilitet. Dette understreker viktigheten av å utforme regional
politikk som tar hensyn til lokale forhold, og som ser økonomisk
utvikling og fordelingshensyn i sammenheng. Videre forskning bør derfor
inkludere flere regioner, lengre tidsserier og et bredere sett av
strukturelle og institusjonelle variabler for å få en mer omfattende
forståelse av hvordan vekst og ulikhet påvirker hverandre over tid.

\subsection{VI. Begrensninger og videre
forskning}\label{vi.-begrensninger-og-videre-forskning}

\paragraph{\texorpdfstring{\textbf{Forskningsmessige
begrensninger}}{Forskningsmessige begrensninger}}\label{forskningsmessige-begrensninger}

Det er flere begrensninger som bør tas i betraktning når funnene fra
denne studien vurderes. For det første er datagrunnlaget relativt lite,
med kun 29 europeiske regioner inkludert i analysen. Et så begrenset
utvalg reduserer den statistiske styrken i modellene og øker risikoen
for at reelle sammenhenger ikke fanges opp. Dette gjelder særlig for
undergruppeanalysene, der datasettet deles opp, og antallet
observasjoner i hver gruppe blir enda mindre.

For det andre bygger hovedanalysene på tverrsnittsdata fra ett enkelt
år. Slike data gir et øyeblikksbilde av sammenhengen mellom vekst og
ulikhet, men gjør det vanskelig å identifisere kausale mekanismer eller
undersøke hvordan forholdet utvikler seg over tid. Selv om Assignment 3
benyttet enkelte panel-lignende modeller, var tidsdimensjonen for
begrenset til å trekke sterke konklusjoner om dynamiske prosesser.

En tredje begrensning er knyttet til måleproblemer og datakvalitet.
Variabler som arbeidsledighet, utdanningsnivå og befolkningstetthet
fanger ikke nødvendigvis opp alle regionale forskjeller som påvirker
ulikhet. Andre faktorer -- som institusjonell kvalitet, sektorstruktur,
migrasjon eller skatt- og overføringssystemer -- er ikke inkludert i
analysen, men kan spille en viktig rolle.

Til slutt kan modellvalgene i seg selv utgjøre en begrensning. Selv om
både lineære, logaritmiske og kvadratiske spesifikasjoner ble testet,
kan mer avanserte modeller -- for eksempel ikke-parametriske
tilnærminger, maskinlæringsmetoder eller instrumentvariabelanalyser --
gi et rikere bilde av sammenhengen mellom vekst og ulikhet.

\paragraph{\texorpdfstring{\textbf{Forslag til videre
forskning}}{Forslag til videre forskning}}\label{forslag-til-videre-forskning}

Fremtidig forskning bør søke å inkludere \textbf{flere land og flere
regioner}, for å bedre fange den institusjonelle og økonomiske
variasjonen som finnes i Europa. Et bredere regionalt datasett ville
gjøre det mulig å trekke sterkere slutninger om hvordan nasjonale
rammebetingelser påvirker vekst--ulikhet-sammenhengen.

Et annet viktig steg vil være å benytte \textbf{lengre tidsserier}, som
muliggjør mer avanserte panelanalyser og bedre identifikasjon av kausale
forhold. Med flere år tilgjengelig kan man undersøke om vekst påvirker
ulikhet i regioner som følger ulike utviklingsbaner, og om effektene er
midlertidige eller varige.

Videre forskning bør også inkludere et bredere sett av regionale
indikatorer. Variabler som næringsstruktur, teknologisk utvikling,
migrasjonsstrømmer, inntekts-mobilitet og institusjonelle kvalitetsmål
kan gi en dypere forståelse av mekanismene som knytter vekst til
ulikhet. Særlig kan forskjeller mellom nord- og sør-europeiske regioner,
eller mellom post-sosialistiske og vestlige økonomier, gi viktige
innsikter.

Et lovende spor er også bruk av \textbf{spatial econometrics},som tar
hensyn til geografisk avhengighet mellom observasjoner, ettersom
regioner ofte påvirker hverandre. For eksempel kan økonomisk vekst i en
region ha fordelingsmessige konsekvenser i omkringliggende regioner.

Til slutt kan videre forskning utforske hvilke politiske tiltak som
bidrar til at økonomisk vekst blir mer inkluderende. Siden resultatene i
denne studien viser at arbeidsledighet er en nøkkelfaktor i hvordan
vekst påvirker ulikhet, kan framtidige analyser vurdere sammenhengen
mellom aktiv arbeidsmarkedspolitikk, kompetanseprogrammer og
fordelingsutfall.

\subsection{VII.Conclusion}\label{vii.conclusion}

• Summary:~Provide a summary of the main findings and their
significance. • Final Reflection:~Briefly reflect on the research
process and the contributions of your study to the field of regional
development and inequality.

\subsection{VIII.References}\label{viii.references}

\phantomsection\label{refs}
\begin{CSLReferences}{1}{0}
\bibitem[\citeproctext]{ref-barro2004}
Barro, R. J., \& Sala-i-Martin, X. (2004). \emph{Economic growth} (2nd
ed). MIT Press.

\bibitem[\citeproctext]{ref-kuznets1955}
Kuznets, S. (1955). Economic growth and income inequality.
\emph{American Economic Review}, \emph{45}(1), 1--28.

\bibitem[\citeproctext]{ref-lessmann2017}
Lessmann, C. (2017). Regional inequality, convergence, and its
determinants -- {A} view from outer space\ding{73}. \emph{European
Economic Review}, 23.

\bibitem[\citeproctext]{ref-mdingi2021}
Mdingi, K., \& Ho, S.-Y. (2021). Literature review on income inequality
and economic growth. \emph{MethodsX}, \emph{8}, 101402.
\url{https://doi.org/10.1016/j.mex.2021.101402}

\bibitem[\citeproctext]{ref-piketty2017}
Piketty, T. (2017). \emph{Capital in the twenty-first century} (A.
Goldhammer, Trans.). The Belknap Press of Harvard University Press.

\end{CSLReferences}

\subsection{Document your use of AI
tools}\label{document-your-use-of-ai-tools}




\end{document}
