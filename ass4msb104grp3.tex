% Options for packages loaded elsewhere
% Options for packages loaded elsewhere
\PassOptionsToPackage{unicode}{hyperref}
\PassOptionsToPackage{hyphens}{url}
\PassOptionsToPackage{dvipsnames,svgnames,x11names}{xcolor}
%
\documentclass[
  letterpaper,
  DIV=11,
  numbers=noendperiod]{scrartcl}
\usepackage{xcolor}
\usepackage{amsmath,amssymb}
\setcounter{secnumdepth}{-\maxdimen} % remove section numbering
\usepackage{iftex}
\ifPDFTeX
  \usepackage[T1]{fontenc}
  \usepackage[utf8]{inputenc}
  \usepackage{textcomp} % provide euro and other symbols
\else % if luatex or xetex
  \usepackage{unicode-math} % this also loads fontspec
  \defaultfontfeatures{Scale=MatchLowercase}
  \defaultfontfeatures[\rmfamily]{Ligatures=TeX,Scale=1}
\fi
\usepackage{lmodern}
\ifPDFTeX\else
  % xetex/luatex font selection
\fi
% Use upquote if available, for straight quotes in verbatim environments
\IfFileExists{upquote.sty}{\usepackage{upquote}}{}
\IfFileExists{microtype.sty}{% use microtype if available
  \usepackage[]{microtype}
  \UseMicrotypeSet[protrusion]{basicmath} % disable protrusion for tt fonts
}{}
\makeatletter
\@ifundefined{KOMAClassName}{% if non-KOMA class
  \IfFileExists{parskip.sty}{%
    \usepackage{parskip}
  }{% else
    \setlength{\parindent}{0pt}
    \setlength{\parskip}{6pt plus 2pt minus 1pt}}
}{% if KOMA class
  \KOMAoptions{parskip=half}}
\makeatother
% Make \paragraph and \subparagraph free-standing
\makeatletter
\ifx\paragraph\undefined\else
  \let\oldparagraph\paragraph
  \renewcommand{\paragraph}{
    \@ifstar
      \xxxParagraphStar
      \xxxParagraphNoStar
  }
  \newcommand{\xxxParagraphStar}[1]{\oldparagraph*{#1}\mbox{}}
  \newcommand{\xxxParagraphNoStar}[1]{\oldparagraph{#1}\mbox{}}
\fi
\ifx\subparagraph\undefined\else
  \let\oldsubparagraph\subparagraph
  \renewcommand{\subparagraph}{
    \@ifstar
      \xxxSubParagraphStar
      \xxxSubParagraphNoStar
  }
  \newcommand{\xxxSubParagraphStar}[1]{\oldsubparagraph*{#1}\mbox{}}
  \newcommand{\xxxSubParagraphNoStar}[1]{\oldsubparagraph{#1}\mbox{}}
\fi
\makeatother

\usepackage{color}
\usepackage{fancyvrb}
\newcommand{\VerbBar}{|}
\newcommand{\VERB}{\Verb[commandchars=\\\{\}]}
\DefineVerbatimEnvironment{Highlighting}{Verbatim}{commandchars=\\\{\}}
% Add ',fontsize=\small' for more characters per line
\usepackage{framed}
\definecolor{shadecolor}{RGB}{241,243,245}
\newenvironment{Shaded}{\begin{snugshade}}{\end{snugshade}}
\newcommand{\AlertTok}[1]{\textcolor[rgb]{0.68,0.00,0.00}{#1}}
\newcommand{\AnnotationTok}[1]{\textcolor[rgb]{0.37,0.37,0.37}{#1}}
\newcommand{\AttributeTok}[1]{\textcolor[rgb]{0.40,0.45,0.13}{#1}}
\newcommand{\BaseNTok}[1]{\textcolor[rgb]{0.68,0.00,0.00}{#1}}
\newcommand{\BuiltInTok}[1]{\textcolor[rgb]{0.00,0.23,0.31}{#1}}
\newcommand{\CharTok}[1]{\textcolor[rgb]{0.13,0.47,0.30}{#1}}
\newcommand{\CommentTok}[1]{\textcolor[rgb]{0.37,0.37,0.37}{#1}}
\newcommand{\CommentVarTok}[1]{\textcolor[rgb]{0.37,0.37,0.37}{\textit{#1}}}
\newcommand{\ConstantTok}[1]{\textcolor[rgb]{0.56,0.35,0.01}{#1}}
\newcommand{\ControlFlowTok}[1]{\textcolor[rgb]{0.00,0.23,0.31}{\textbf{#1}}}
\newcommand{\DataTypeTok}[1]{\textcolor[rgb]{0.68,0.00,0.00}{#1}}
\newcommand{\DecValTok}[1]{\textcolor[rgb]{0.68,0.00,0.00}{#1}}
\newcommand{\DocumentationTok}[1]{\textcolor[rgb]{0.37,0.37,0.37}{\textit{#1}}}
\newcommand{\ErrorTok}[1]{\textcolor[rgb]{0.68,0.00,0.00}{#1}}
\newcommand{\ExtensionTok}[1]{\textcolor[rgb]{0.00,0.23,0.31}{#1}}
\newcommand{\FloatTok}[1]{\textcolor[rgb]{0.68,0.00,0.00}{#1}}
\newcommand{\FunctionTok}[1]{\textcolor[rgb]{0.28,0.35,0.67}{#1}}
\newcommand{\ImportTok}[1]{\textcolor[rgb]{0.00,0.46,0.62}{#1}}
\newcommand{\InformationTok}[1]{\textcolor[rgb]{0.37,0.37,0.37}{#1}}
\newcommand{\KeywordTok}[1]{\textcolor[rgb]{0.00,0.23,0.31}{\textbf{#1}}}
\newcommand{\NormalTok}[1]{\textcolor[rgb]{0.00,0.23,0.31}{#1}}
\newcommand{\OperatorTok}[1]{\textcolor[rgb]{0.37,0.37,0.37}{#1}}
\newcommand{\OtherTok}[1]{\textcolor[rgb]{0.00,0.23,0.31}{#1}}
\newcommand{\PreprocessorTok}[1]{\textcolor[rgb]{0.68,0.00,0.00}{#1}}
\newcommand{\RegionMarkerTok}[1]{\textcolor[rgb]{0.00,0.23,0.31}{#1}}
\newcommand{\SpecialCharTok}[1]{\textcolor[rgb]{0.37,0.37,0.37}{#1}}
\newcommand{\SpecialStringTok}[1]{\textcolor[rgb]{0.13,0.47,0.30}{#1}}
\newcommand{\StringTok}[1]{\textcolor[rgb]{0.13,0.47,0.30}{#1}}
\newcommand{\VariableTok}[1]{\textcolor[rgb]{0.07,0.07,0.07}{#1}}
\newcommand{\VerbatimStringTok}[1]{\textcolor[rgb]{0.13,0.47,0.30}{#1}}
\newcommand{\WarningTok}[1]{\textcolor[rgb]{0.37,0.37,0.37}{\textit{#1}}}

\usepackage{longtable,booktabs,array}
\usepackage{calc} % for calculating minipage widths
% Correct order of tables after \paragraph or \subparagraph
\usepackage{etoolbox}
\makeatletter
\patchcmd\longtable{\par}{\if@noskipsec\mbox{}\fi\par}{}{}
\makeatother
% Allow footnotes in longtable head/foot
\IfFileExists{footnotehyper.sty}{\usepackage{footnotehyper}}{\usepackage{footnote}}
\makesavenoteenv{longtable}
\usepackage{graphicx}
\makeatletter
\newsavebox\pandoc@box
\newcommand*\pandocbounded[1]{% scales image to fit in text height/width
  \sbox\pandoc@box{#1}%
  \Gscale@div\@tempa{\textheight}{\dimexpr\ht\pandoc@box+\dp\pandoc@box\relax}%
  \Gscale@div\@tempb{\linewidth}{\wd\pandoc@box}%
  \ifdim\@tempb\p@<\@tempa\p@\let\@tempa\@tempb\fi% select the smaller of both
  \ifdim\@tempa\p@<\p@\scalebox{\@tempa}{\usebox\pandoc@box}%
  \else\usebox{\pandoc@box}%
  \fi%
}
% Set default figure placement to htbp
\def\fps@figure{htbp}
\makeatother


% definitions for citeproc citations
\NewDocumentCommand\citeproctext{}{}
\NewDocumentCommand\citeproc{mm}{%
  \begingroup\def\citeproctext{#2}\cite{#1}\endgroup}
\makeatletter
 % allow citations to break across lines
 \let\@cite@ofmt\@firstofone
 % avoid brackets around text for \cite:
 \def\@biblabel#1{}
 \def\@cite#1#2{{#1\if@tempswa , #2\fi}}
\makeatother
\newlength{\cslhangindent}
\setlength{\cslhangindent}{1.5em}
\newlength{\csllabelwidth}
\setlength{\csllabelwidth}{3em}
\newenvironment{CSLReferences}[2] % #1 hanging-indent, #2 entry-spacing
 {\begin{list}{}{%
  \setlength{\itemindent}{0pt}
  \setlength{\leftmargin}{0pt}
  \setlength{\parsep}{0pt}
  % turn on hanging indent if param 1 is 1
  \ifodd #1
   \setlength{\leftmargin}{\cslhangindent}
   \setlength{\itemindent}{-1\cslhangindent}
  \fi
  % set entry spacing
  \setlength{\itemsep}{#2\baselineskip}}}
 {\end{list}}
\usepackage{calc}
\newcommand{\CSLBlock}[1]{\hfill\break\parbox[t]{\linewidth}{\strut\ignorespaces#1\strut}}
\newcommand{\CSLLeftMargin}[1]{\parbox[t]{\csllabelwidth}{\strut#1\strut}}
\newcommand{\CSLRightInline}[1]{\parbox[t]{\linewidth - \csllabelwidth}{\strut#1\strut}}
\newcommand{\CSLIndent}[1]{\hspace{\cslhangindent}#1}



\setlength{\emergencystretch}{3em} % prevent overfull lines

\providecommand{\tightlist}{%
  \setlength{\itemsep}{0pt}\setlength{\parskip}{0pt}}



 


\KOMAoption{captions}{tableheading}
\makeatletter
\@ifpackageloaded{caption}{}{\usepackage{caption}}
\AtBeginDocument{%
\ifdefined\contentsname
  \renewcommand*\contentsname{Table of contents}
\else
  \newcommand\contentsname{Table of contents}
\fi
\ifdefined\listfigurename
  \renewcommand*\listfigurename{List of Figures}
\else
  \newcommand\listfigurename{List of Figures}
\fi
\ifdefined\listtablename
  \renewcommand*\listtablename{List of Tables}
\else
  \newcommand\listtablename{List of Tables}
\fi
\ifdefined\figurename
  \renewcommand*\figurename{Figure}
\else
  \newcommand\figurename{Figure}
\fi
\ifdefined\tablename
  \renewcommand*\tablename{Table}
\else
  \newcommand\tablename{Table}
\fi
}
\@ifpackageloaded{float}{}{\usepackage{float}}
\floatstyle{ruled}
\@ifundefined{c@chapter}{\newfloat{codelisting}{h}{lop}}{\newfloat{codelisting}{h}{lop}[chapter]}
\floatname{codelisting}{Listing}
\newcommand*\listoflistings{\listof{codelisting}{List of Listings}}
\makeatother
\makeatletter
\makeatother
\makeatletter
\@ifpackageloaded{caption}{}{\usepackage{caption}}
\@ifpackageloaded{subcaption}{}{\usepackage{subcaption}}
\makeatother
\usepackage{bookmark}
\IfFileExists{xurl.sty}{\usepackage{xurl}}{} % add URL line breaks if available
\urlstyle{same}
\hypersetup{
  pdftitle={Økonomisk vekst og regional ulikhet: En empirisk analyse av sammenhengen mellom vekst og Gini-koeffisient på tvers av regioner},
  pdfauthor={Irjan \& Magnus},
  colorlinks=true,
  linkcolor={blue},
  filecolor={Maroon},
  citecolor={Blue},
  urlcolor={Blue},
  pdfcreator={LaTeX via pandoc}}


\title{Økonomisk vekst og regional ulikhet: En empirisk analyse av
sammenhengen mellom vekst og Gini-koeffisient på tvers av regioner}
\author{Irjan \& Magnus}
\date{2025-01-01}
\begin{document}
\maketitle


\subsection{}\label{section}

Irjan tar oddetall. Magnum opus tar partall

\begin{Shaded}
\begin{Highlighting}[]
\FunctionTok{Sys.getlocale}\NormalTok{()}
\end{Highlighting}
\end{Shaded}

\begin{verbatim}
[1] "LC_COLLATE=Norwegian_Norway.1252;LC_CTYPE=Norwegian_Norway.1252;LC_MONETARY=Norwegian_Norway.1252;LC_NUMERIC=C;LC_TIME=Norwegian_Norway.1252"
\end{verbatim}

\subsection{Objective}\label{objective}

\subsection{Synthesize and consolidate the critical findings from
Assignments 1, 2, and 3 into a
coherent}\label{synthesize-and-consolidate-the-critical-findings-from-assignments-1-2-and-3-into-a-coherent}

academic paper. Max size 20 pages.

\subsection{0. Short abstract}\label{short-abstract}

\subsection{I. Introduction}\label{i.-introduction}

• Background:~ Provide a brief background on the topics of regional
development and inequality.

Objectives:~Outline the main objectives and research questions addressed
in Assignments 1, 2, and 3. • Significance:~Highlight the importance and
relevance of the research undertaken.

\subsection{II.Literaturgjennomgang}\label{ii.literaturgjennomgang}

(Piketty, 2017) Previous Work:~Summarize key literature related to
regional development and inequality. ‣ Utilize the RStudio citation
tools in your summary of the literature. • Research Gap:~Identify the
research gap your assignments aim to address.

Forholdet mellom økonomisk vekst og inntektsulikhet har lenge vært et
sentralt og omdiskutert tema i økonomisk teori. Den klassiske
Kuznets-hypotesen (Kuznets, 1955) foreslår en omvendt U-formet kurve
hvor ulikheten stiger i tidlige utviklingsfaser for så å falle når
økonomien modnes. Selv om dette rammeverket har sterk historisk
innflytelse, har senere forskning vist at mekanismene bak vekst og
ulikhet er langt mer komplekse, og at sammenhengen varierer betydelig
mellom regioner og tidsperioder (Piketty, 2017).

En viktig utvidelse kommer fra moderne regionaløkonomisk teori, særlig
forskning på konvergens og regional divergens. Barro og Sala-i-Martin
(1995) argumenterer for at regioner kan nærme seg hverandre økonomisk
(konvergere) dersom kapital og arbeidskraft flyter fritt, men at
ulikheter kan vedvare når særtrekk som utdanningsnivå, innovasjonsevne
eller næringsstruktur skiller regionene.

Et sentralt bidrag til denne litteraturen er Lessmann (2017), som
undersøker regional ulikhet og konvergens i et globalt perspektiv ved
hjelp av satellittdata. Denne studien viser at regional ulikhet ikke
nødvendigvis avtar i takt med økonomisk utvikling, og at forskjeller i
institusjoner, politisk stabilitet og geografiske forhold spiller en
langt større rolle enn tidligere antatt. Lessmann finner også at
regioner med svakere styringskvalitet og lavere humankapitalutvikling
har en tendens til å oppleve mer vedvarende eller økende ulikhet. Dette
støtter tanken om at vekst alene ikke er tilstrekkelig for å redusere
ulikhet uten at underliggende strukturelle faktorer tas i betraktning.

Agglomerasjonseffekter fremhevet av Moretti (2012) peker i samme
retning: regioner med høy kompetanse og teknologisk innovasjon
tiltrekker seg kapital og arbeidskraft, noe som forsterker geografisk
konsentrasjon av velstand. Dette kan øke ulikheten mellom regioner selv
i perioder med høy samlet økonomisk vekst.

\subsection{III Data and Methodology}\label{iii-data-and-methodology}

• Data Sources:~Briefly describe the primary data sources used in the
previous assignments, noting any limitations. • Methodological
Approach:~Summarize the methodologies applied in Assignments 1, 2, and
3, highlighting the rationale behind choosing each method.

\subsection{IV.Empirical Findings}\label{iv.empirical-findings}

• Cross-sectional Estimates:~Summarize essential results obtained from
the cross-sectional analysis in Assignment 2. • Alternative Functional
Forms and Panel Estimates:~Present the key insights derived from
exploring alternative functional forms and panel estimates in Assignment
3.

\subsection{V.Discussion}\label{v.discussion}

• Key Insights:~Synthesize and discuss the central insights emerging
from the empirical findings. • Policy Implications:~Deliberate on the
potential implications of your results for policymaking and practice in
the realm of regional development and inequality.

\subsection{VI.Limitations and Future
Research}\label{vi.limitations-and-future-research}

• Research Limitations:~Acknowledge any constraints or limitations
encountered during your analysis.

\subsection{VII.Conclusion}\label{vii.conclusion}

• Summary:~Provide a summary of the main findings and their
significance. • Final Reflection:~Briefly reflect on the research
process and the contributions of your study to the field of regional
development and inequality.

\subsection{VIII.References •}\label{viii.references}

Include a list of all references cited in the paper, formatted according
to a standard academic style. Use APA7, you will find apa7.csl under
Filer at the Canvas site of the course. Include it with csl: apa7.csl in
your YAML header. Appendix (Kuznets, 1955)

\subsection*{Document your use of AI
tools}\label{document-your-use-of-ai-tools}
\addcontentsline{toc}{subsection}{Document your use of AI tools}

\phantomsection\label{refs}
\begin{CSLReferences}{1}{0}
\bibitem[\citeproctext]{ref-kuznets1955}
Kuznets, S. (1955). Economic growth and income inequality.
\emph{American Economic Review}, \emph{45}(1), 1--28.

\bibitem[\citeproctext]{ref-lessmann2017}
Lessmann, C. (2017). Regional inequality, convergence, and its
determinants -- {A} view from outer space\ding{73}. \emph{European
Economic Review}, 23.

\bibitem[\citeproctext]{ref-piketty2017}
Piketty, T. (2017). \emph{Capital in the twenty-first century} (A.
Goldhammer, Trans.). The Belknap Press of Harvard University Press.

\end{CSLReferences}




\end{document}
